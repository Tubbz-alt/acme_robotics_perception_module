\href{https://travis-ci.org/nr-parikh/acme_robotics_perception_module}{\tt } \href{https://coveralls.io/github/nr-parikh/acme_robotics_perception_module?branch=master}{\tt } \href{https://github.com/nr-parikh/acme_robotics_perception_module/blob/master/LICENSE}{\tt }

\subsection*{Overview}

This is a naive perception module for the Unmanned Aerial Vehicle by Acme Robotics. The new robot being developed by Acme robotics has a downward facing camera and a front facing ultrasonic sensor. The task of the robot is to detect the boundary lines which might be there on the ground and head in the direction opposite to that of the boundary lines. It should simultaneously also monitor the incoming reading from the ultrasonic sensor and avoid obstacles if any. The developed perception module has the capability of combining the information from different sensors and decide the heading direction. The details about the module are discussed in detail here.

The dependencies of the module are\+:
\begin{DoxyItemize}
\item Open\+CV 3.\+2 or higher\+: To download and build the library, please follow the instructions given \href{https://www.learnopencv.com/install-opencv3-on-ubuntu/}{\tt here}.
\end{DoxyItemize}

\subsection*{Instructions to build the module}

Before building the module, please ensure that the dependencies are met. Please follow the instructions given below to build the module.


\begin{DoxyCode}
1 $ git clone https://github.com/nr-parikh/acme\_robotics\_perception\_module.git
2 $ cd <path to repository>
3 $ mkdir build
4 $ cd build
5 $ cmake ..
6 $ make
\end{DoxyCode}


Once the module is built correctly, to run the executable type the following command\+: 
\begin{DoxyCode}
1 $ cd <build folder of the module>
2 $ ./perception\_module/shell-app
\end{DoxyCode}


\subsection*{Solo Iterative Process}

While developing this module S\+IP process was followed. The link to the S\+IP sheet for this module is \href{https://docs.google.com/spreadsheets/d/1xeEtkg9tZwtrnPBAMPW0ByWKBNMoUEidjYDhUrIFllk/edit?usp=sharing}{\tt here}. Please note that the legend of the code is mentioned in the sheet itself.

Release backlog shows the improvements in the expected time to complete a certain task as the module is developed. The task backlog given detailed breakdown of time taken for each individual tasks.

\subsection*{Sensors}

This module relies on the information incoming from two different sensors. The sensors which it uses are\+:
\begin{DoxyItemize}
\item R\+GB camera
\item Ultrasonic \hyperlink{class_sensor}{Sensor}
\end{DoxyItemize}

The camera is downward facing. The position of the camera is such that it becomes easier for it to get the information of the things on the ground. The module creates a camera object which uses Open\+CV\textquotesingle{}s {\itshape Video\+Capture} oject to reda frames of the incoming video stream. If there is a need for debugging, the camera has debugging mode. A flag can be used to toggle the debug mode. In the debug mode, the camera reads a default image provided in the {\itshape data} folder. The camera module has {\itshape is\+\_\+running\+\_\+} flag which indicates whether the camera is running or not.

Ultrasonic sensor is front facing. The location is such that it can detect the obstacles in the front of the robot. In this module the readings from the sensor are simulated using pseudo-\/random number generator. But the logic is treats it as sensor readings and not as random numbers. The ultrasonic sensor also has {\itshape is\+\_\+running\+\_\+} flag which indicates whether the sensor is running or not.

\subsection*{Algorithm}

The task at hand for robot is to be within the boundary it sees on the ground while avoiding the obstacles it sees in front of it. The main task for the perception module is to detect the caution tape in the image. To detect the tape a particular feature of the caution tape, the fact that it is yello and black, is used. Since, the colors of the tape are already known it becomes easier to detect the tape. {\itshape \hyperlink{class_camera}{Camera}} module pre-\/processes the image. Pre-\/processing includes finding the {\itshape in\+Range} pixels i.\+e. find the pixels which are within the range of {\itshape yellow} and {\itshape black} colors. Adding the two images will give the {\itshape blob} of tape in the image. The {\itshape perception} module then uses this pre-\/processed image as an input to determine the location of the tape in it. To locate the line in the given module has 2 methods\+:


\begin{DoxyItemize}
\item First is to {\itshape Hough transforms} to find the lines. Open\+CV\textquotesingle{}s {\itshape Houghlines} function finds the possible lines in the image and sorts them according the number of votes. The highest voted line is the first element of the vector which can be used to find the points on that line.
\item Second method is to find the largest closed contour in the image which is probably the tape. To detect the largest contour, the simplest way is to iterate over all possible closed contours and find the largest contour based on its area. The center of the largest contour can then be found to decide the heading direction.
\end{DoxyItemize}

Based on the method used to find the location of the tape, {\itshape control} module has different methods to compute control action. For the first method described above, the location of both the points can be used to determine whether the line is horizontal or vertical and then check if the points are located in which quadrant of the image. The angle of heading can also be calculated from the {\itshape slope} of the line. For the second method, the location of the center can be used to approximately determine the quadrant in which the line lies. This method is not good since it can locate the tape approximately. Hence, the usage of the first method is suggested and is set as default.

The {\itshape control} module first checks the reading from the ultrasonic sensor to determine whether any obstacle is present in front of the robot or not. If it finds an obstacle it issues the command to {\itshape go back}. If no obstacles are found then it detects the tape in the image. It then locates the points and based on the location of points it generates the control action.

\subsection*{Output}

The following table shows the output for the vertically detected line in the image.

\tabulinesep=1mm
\begin{longtabu} spread 0pt [c]{*2{|X[-1]}|}
\hline
\rowcolor{\tableheadbgcolor}{\bf Input Image }&{\bf Pre-\/processed output  }\\\cline{1-2}
\endfirsthead
\hline
\endfoot
\hline
\rowcolor{\tableheadbgcolor}{\bf Input Image }&{\bf Pre-\/processed output  }\\\cline{1-2}
\endhead
 & \\\cline{1-2}
\end{longtabu}


\tabulinesep=1mm
\begin{longtabu} spread 0pt [c]{*2{|X[-1]}|}
\hline
\rowcolor{\tableheadbgcolor}{\bf Detected Line }&{\bf Generated output action  }\\\cline{1-2}
\endfirsthead
\hline
\endfoot
\hline
\rowcolor{\tableheadbgcolor}{\bf Detected Line }&{\bf Generated output action  }\\\cline{1-2}
\endhead
 & \\\cline{1-2}
\end{longtabu}


The table below shows the output for the horizontally detected line in the image.

\tabulinesep=1mm
\begin{longtabu} spread 0pt [c]{*2{|X[-1]}|}
\hline
\rowcolor{\tableheadbgcolor}{\bf Input Image }&{\bf Pre-\/processed output  }\\\cline{1-2}
\endfirsthead
\hline
\endfoot
\hline
\rowcolor{\tableheadbgcolor}{\bf Input Image }&{\bf Pre-\/processed output  }\\\cline{1-2}
\endhead
 & \\\cline{1-2}
\end{longtabu}


\tabulinesep=1mm
\begin{longtabu} spread 0pt [c]{*2{|X[-1]}|}
\hline
\rowcolor{\tableheadbgcolor}{\bf Detected Line }&{\bf Generated output action  }\\\cline{1-2}
\endfirsthead
\hline
\endfoot
\hline
\rowcolor{\tableheadbgcolor}{\bf Detected Line }&{\bf Generated output action  }\\\cline{1-2}
\endhead
 & \\\cline{1-2}
\end{longtabu}
